
\section{Introduction}

% --------------------
%\begin{itemize}
%\item  Books: \cite{xiao2010underwater, otnes2012underwater}
%
%\item  Surveys: \cite{akyildiz2005underwater, partan2007survey,
%       climent2014underwater, heidemann2012underwater}
%
%\item  Architecture: \cite{jaffe2006sensor}
%
%\item  Real world: \cite{le2013realWorld, roy2006wideArea}
%\item  Coverage: \cite{senel2013auto, pompili:2006deploymen}
%
%\item  Reliability: \cite{Co87}
%\item  Delays: \cite{pu2013comparing}
%
%\item  Very related: \cite{bower1989evidence, bower91simple,
%       caruso2008meandering, luo2009double, elmallah01supporting}
%\end{itemize}
% --------------------

Research on Underwater Sensor Networks (UWSNs) has intensified in recent
years.
Interest in such research has been fueled by many important underwater
sensing applications and services that can be supported by such networks
(see, e.g., \cite{akyildiz2005underwater,heidemann2012underwater}).
%
The domains of such applications are diverse and can be roughly classified as
scientific applications (e.g., collecting data on geological processes,
analyzing water characteristics, study of marine life),
industrial applications (e.g., monitoring underwater equipment and pipelines
used in oil industry),
humanitarian applications (e.g., search and survey missions, prediction of
natural disturbances) and/or
military and home land security applications (e.g., securing port facilities).

To serve the above diverse types of applications, various types
of UWSN deployments are used.
%
In \cite{heidemann2012underwater}, for example, UWSN deployments are
classified as being either static, semi-mobile, or mobile.
%
Static networks have nodes attached to underwater ground,
anchored buoys, or docks.
%
Semi-mobile networks may have collection of nodes attached to a
free floating buoy. Nodes in semi-mobile networks are subject to small scale
movements.
%
Mobile networks may be composed of drifters with no self mobility
capability, or nodes with mobility capability. Nodes in such networks
are subject to large scale movements.
%
UWSN deployments may occur over many short periods of times (e.g.,
several days at a time), so as to conduct several missions over
a large area of interest.
%

Due to the importance of such applications, and the challenges encountered
in their design, extensive work spanning all five layers of the Internet
protocol stack appear in the literature.
%
See, e.g., the chapters in \cite{xiao2010underwater, otnes2012underwater},
and the survey articles in \cite{partan2007survey, climent2014underwater}.
%
Examples of real-world UWSN work include
\cite{rice2000evolution,jaffe2006sensor,roy2006wideArea,rice2007seaweb,
pu2013comparing}.

% ------------------------------
In this paper, we consider semi-mobile and mobile networks.
%
Maintaining connectivity in such networks is a crucial aspect for
any task requiring node collaboration.
%
Our interest in on developing methodologies that allow a designer
to analyze the likelihood that a network (or part of it) is connected
at a given time interval.

For literature review, we note that several experimental and analytical
results in oceanography literature have shaped our current understanding
of mobility for underwater sensor networks.
%
Of the vast literature existing in the field, we recall the following
early landmark results.
%
In \cite{bower1989evidence}, the authors report on several observations
collected in the Gulf Stream using thirty-seven RAFOS drifters launched
off Cape Hatteras. Mobility of the free floating drifters are tracked
for 30 or 45 days.
%
In \cite{bower91simple}, the author describes a 2-dimensional kinematic
model of a meandering jet. 
%
The model captures the striking patterns of cross-stream and vertical
motion associated with meanders observed in \cite{bower1989evidence}.


Investigations and results obtained in the above directions have
been valuable for networking researchers approaching the challenge
of modelling the mobility of underwater sensor networks.
%
In \cite{caruso2008meandering}, the authors adopt a kinematic model
for capturing the effect of meandering sub-surface currents and v̰o̰r̰t̰ḭc̰ḛs̰
of free floating sensor nodes.
%
The model, called the meandering current mobility model, is useful
for large coastal environments that span several kilometers.
%
It captures the strong correlations in mobility of nearby sensor nodes.
%
Using simulation, the authors investigate several network connectivity,
coverage, and localization aspects.


In \cite{luo2009double}, the authors consider sensor nodes with movement
capability.
%
Each node incurs both uncontrollable and controllable mobility
(abbreviated as U-mobility and C-mobility).
%
Using a grid layout that divides a geographic area into cells, the authors
adopt a probabilistic U-mobility model.
%
The model takes into consideration two types of effects:
{\em local variety} effects (caused by reefs, turbulence), and
{main circulation}  effects (caused by wind, salinity).
%
Using such model, the authors present an energy efficient approach for
satisfying network coverage requirements.


In this paper, we adopt a simple {\em probabilistic locality} model
where the geographic area under consideration is partitioned into
disjoint regions (rectangles) using a grid layout.
%
In our model, the use of a high grid resolution (i.e., a layout with small
regions) can potentially give accurate results at the expense of decreased
solution efficiency.
%
We assume that one can utilize a physical model of underwater currents to
compute the probability that a sensor node is located at a given region
during some time interval of interest.
%
For example, one may utilize the kinematic model adopted in
\cite{caruso2008meandering} to compute such probabilities from a
sufficiently large number of node trajectories generated by the model.
%
Using such probability distribution, one obtains a {\em probabilistic} graph
model of the network.

Our work here formalizes two problems, denoted $\ACONN$ and $\SCONN$,
that call for determining the likelihood that a probabilistic graph
is entirely or partially connected.
%
Our main contribution is an efficient dynamic programming algorithm to
solve both problems on tree-like networks.
%
The algorithm can be used to derive lower bounds on the solution of
any arbitrary probabilistic network.
%
Our devised algorithm extends a result in \cite{elmallah01supporting}
to compute the probability that a given sequence of nodes 
in a probabilistic network forms a simple connected path.
%
To the best of our knowledge, both the problems formulation and devised
algorithm are novel aspects of the paper.  

In the next two sections we outline the system model and problem formulations.
Section 4 then gives a detailed description of the algorithm.
